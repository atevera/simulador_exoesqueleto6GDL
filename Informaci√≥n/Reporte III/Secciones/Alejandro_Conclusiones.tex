\subsection{Tevera Ruiz Alejandro}
    \noindent Se alcanzó el objetivo deseado orientado al desarrollo del simulador dinámico
    para el exoesqueleto de 6GdL (\emph{sistema no conservativo}) con base al diseño de \cite{hexotrac}. Para ello, 
    se consideró fundamentos científicos basados principalmente en el modelado de sistemas 
    dinámicos mediante el análisis energético bajo la formulación de \emph{D'Alambert-Lagrange}. 

    De esta manera, se simplificó el proceso de modelado del sistema multipartículas otorgando la posibilidad
    de analizar adecuadamente las variables de estado y el comportamiento energético. Para ello, se construyó 
    el simulador mediante bloques en \emph{Simulink} con base en funciones desarrolladas en \emph{MATLAB}, así como la
    implementación de una interfaz gráfica y visualizador con el objetivo de proveer un entorno de análisis potente pero
    sencillo de configurar bajo ciertos parámetros. 

    Por otro lado, se realizaron diversas pruebas para el desarrollo y la validación del simulador dinámico. Concluyendo 
    así con tres casos de estudio expuestos en la Sección \ref{resultados}. Para el primero y segundo de ellos, el comportamiento
    del \emph{sistema no conservativo} fue el esperado, donde la interacción con las \emph{fuerzas generalizadas} de disipación y/o
    exógenas es notoria.
    
    Sin embargo, los resultados del \emph{sistema conservativo} fueron inconsistentes a la teoría como consecuencia del tipo de solucionador
    utilizado para resolver la dinámica del sistema. Esto puede observarse en el error acumulado de la energía cinética ya que 
    presenta un comportamiento ascendente durante cierto periodo hasta que el solucionador busca corregir el error.
    
    Aunque el $ode15s$ utiliza un método numérico potente para \emph{sistemas rígidos}, el número de operaciones para la integración
    de las variables de estado es alto, por lo que aún realizando el ajuste de la frecuencia de muestreo del solucionador no se presenta
    una mejora significativa, únicamente mayor tiempo requerido para la convergencia. 

    Esto abre la posibilidad de realizar una segunda versión del simulador orientado al modelado dinámico basado en \emph{BDA} con el
    fin de reducir el consumo de recursos informáticos y por ende, las operaciones aritméticas para la solución de la dinámica directa
    del sistema.