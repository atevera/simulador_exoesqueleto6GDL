\subsection{García Álvarez Gregorio Eliezer}
\noindent Se alcanzó el objetivo de este reporte  puesto que se efectuó el 
simulador dinámico, así como su comportamiento semejante a lo esperado en 
la realidad, en el caso del sistema no conservativo.

En contraparte en el sistema que no considera la fricción viscosa, se 
obtuvieron resultados no esperados que se analizaron y discutieron a 
profundidad, con el objetivo de identificar el error y así implementar 
una solución adecuada. 

Se concluye que la discrepancia es debida al integrador. Una hipótesis 
es la siguiente:

Debido a que sobre el sistema solo actúa la fuerza de gravedad, al 
soltarse el exoesqueleto de su posición vertical de "casa", se empieza 
a acelerar y a partir del segundo 5 aún no alcanza una velocidad 
constante porque el sistema se puede considerar como un péndulo 
compuesto de 7 eslabones, lo que conlleva a un sistema caótico 
(sin embargo continúa siendo determinista). Es importante remarcar 
que el 6to referencial presenta una velocidad mayor al del resto 
de los eslabones, porque al ser tan pequeño en relación a los demás, 
implica que tiene el menor momento de inercia. Al ser el último 
eslabón el que mayor velocidad tiene, este influye considerablemente 
en la trayectoria de la cadena cinemática. Y estos cambios de 
aceleraciones sobrepasan al tiempo de procesamiento del integrador,  
acumulando un error numérico y mostrando datos imposibles (crear energía).
