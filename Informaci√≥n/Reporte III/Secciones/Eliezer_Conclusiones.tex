\subsection{García Álvarez Gregorio Eliezer}
\noindent Se alcanzó el objetivo de este reporte  puesto que se efectuó el simulador dinámico, así como su comportamiento
semejante a lo esperado en la realidad, en el caso del sistema no conservativo.
En el caso del sistema sin considerar la fricción viscosa, se obtuvieron resultados no esperados que se analizaron
y discutieron a profundidad, con el objetivo de identificar el error para poder implementar una solución adecuada. 
Se concluye que la discrepancia es debida al integrador. 

Una hipótesis es la siguiente:
Debido a que sobre el sistema solo actúa la fuerza de gravedad, al soltarse el exoesqueleto de su posición vertical
de "casa", se empieza a acelerar y a partir del segundo 5 aún no alcanza una velocidad constante porque el sistema
se puede considerar como un péndulo compuesto de 7 eslabones, lo que conlleva a un sistema caótico (sin embargo
continúa siendo determinista). Es importante remarcar que el 6to referencial presenta una velocidad mayor al del
resto de los eslabones, porque al ser tan pequeño en relación a los demás, implica que tiene el menor momento de
inercia. Al ser el último eslabón el que mayor velocidad tiene, este influye considerablemente en la trayectoria de
la cadena cinemática. Y estos cambios de aceleraciones sobrepasan al tiempo de procesamiento del integrador, 
acumulando un error numérico y mostrando datos imposibles (crear energía).

Por otro lado, un simulador tiene como objetivo experimentar en un entorno virtual, así como ayudar a comprobar
hipótesis. Un ejercicio que se me ocurrió, puede ser el de ejemplificar la teoría del caos, en el caso de un sistema
no conservativo y ceros torques, debido a que el exoesqueleto en estas condiciones se comporta como un péndulo
compuesto. Finalmente, en la animación y las gráficas de posiciones generalizadas del caso 2, observé un cambio de 
giro; esto lo atribuyo al teorema del eje intermedio, sin embargo con los actuales datos no puedo asegurar que este
sucediendo esto, en este momento solo me limito a una interpretación de las gráficas y de lo visualizado en la
animación.