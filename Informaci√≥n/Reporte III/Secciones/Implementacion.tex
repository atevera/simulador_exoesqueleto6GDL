\subsection{Jacobiana} %Revisar si es el nombre correcto 
Como se mencionó anteriormente, la matriz jacobiana se compone de la velocidad lineal y de la velocidad angular, a
fin de obtener la matriz
\begin{equation*}
    \hspace{-10mm}
    J = \left[
        \begin{array}{ccccc}
            J_{1,1} & J_{1,2} & J_{1,3} & J_{1,4} & J_{1,5}\\
            J_{2,1} & J_{2,2} & J_{2,3} & J_{2,4} & J_{2,5}\\
            J_{3,1} & J_{3,2} & J_{3,3} & J_{3,4} & J_{3,5}\\
            J_{4,1} & J_{4,2} & J_{4,3} & J_{4,4} & J_{4,5}\\
            J_{5,1} & J_{5,2} & J_{5,3} & J_{5,4} & J_{5,5}\\
            J_{6,1} & J_{6,2} & J_{6,3} & J_{6,4} & J_{6,5}\\
        \end{array} \right]
    \hspace{10mm}
\end{equation*}
Como se observa la matriz obtenida es una matriz de 6 x 6, al ser una matriz cuadrada, el rango de la matriz es de 6
Para la jacobiana de la velocidad lineal se obtiene: 
\begin{equation*}
    \hspace{-10mm}
    J = \left[
        \begin{array}{ccccc}
            J_{v_1 ,1} & J_{v_1 ,2} & J_{v_1 ,3} & J_{v_1 ,4} & J_{v_1 ,5}\\
            J_{v_2 ,2} & J_{v_2 ,2} & J_{v_2 ,3} & J_{v_2 ,4} & J_{v_2 ,5}\\
            J_{v_3 ,3 } & J_{v_3,2} & J_{v_3 ,3} & J_{v_3 ,4} & J_{v_3 ,5}\\
        \end{array}\right] \hspace{10mm}
\end{equation*} 
\noindent Tomando en cuenta que cada $C_i$ corresponde a $\cos \theta_i$ y cada $S_i$ a $\sin \theta_i$, cada elemento
$J_{v_ij}$ de la matriz es representado por:


Y para la jacobiana de la velocidad angular se tiene:
\begin{equation*}
    \hspace{-10mm}
    J = \left[
        \begin{array}{ccccc}
            J_{\omega_1 ,1} & J_{\omega1 ,2} & J_{\omega1 ,3} & J_{\omega1 ,4} & J_{\omega1 ,5}\\
            J_{\omega2 ,2} & J_{\omega2 ,2} & J_{\omega2 ,3} & J_{\omega2 ,4} & J_{\omega2 ,5}\\
            J_{\omega3 ,3 } & J_{\omega3,2} & J_{\omega3 ,3} & J_{\omega3 ,4} & J_{\omega3 ,5}\\
        \end{array}\right] 
        \hspace{10mm}
\end{equation*} 
\noindent Tomando en cuenta que cada $C_i$ corresponde a $\cos \theta_i$ y cada $S_i$ a $\sin \theta_i$, cada elemento
$J_{\omega_{ij}}$ de la matriz es representado por:
\begin{flushleft}
    \(J_{\omega_1 ,1} = 0 \)\\ \vspace{0.25cm}
    \(J_{\omega1 ,2} = 0\) \\ \vspace{0.25cm}
    \(J_{\omega1 ,3} = 0\) \\ \vspace{0.25cm}
    \(J_{\omega1 ,4} = 0\) \\ \vspace{0.25cm}
    \(J_{\omega1 ,5} = 0\) \\ \vspace{0.25cm}
\end{flushleft}

\begin{flushleft}
    \(J_{\omega2 ,2} = 0 \) \\ \vspace{0.25cm}
    \(J_{\omega2 ,2} = 0\) \\ \vspace{0.25cm}
    \(J_{\omega2 ,3} =0\) \\ \vspace{0.25cm}
    \(J_{\omega2 ,4} =0\) \\ \vspace{0.25cm}
    \(J_{\omega2 ,5 } =0\) \\ \vspace{0.25cm}
\end{flushleft}

\begin{flushleft}
    \(J_{\omega3 ,3} = 0 \) \\ \vspace{0.25cm}
    \(J_{\omega3, 2} = 0 \) \\ \vspace{0.25cm}
    \(J_{\omega3 ,3} = 0 \) \\ \vspace{0.25cm}
    \(J_{\omega3 ,4} = 0 \) \\ \vspace{0.25cm}
    \(J_{\omega3 ,5} = 0\) \\ \vspace{0.25cm}
\end{flushleft}    

\subsection{Dinámica}
Se obtuvieron el vector de Coriolis, Inercia así como el vector de gravedad (el proceso y los resultados obtenidos se
muestran más adelante). A fin de obtener la dinámica del sistema.
Cabe resaltar que para obtener las masas, los centros de masa, así como los tensores de inercia, estos fueron obtenidos
del modelo CAD, los datos se describen a continuación: 
Masas:
\begin{equation*}
    M = \begin{bmatrix} 0 & 0 & 0 & 0 & 0 & 0 \end{bmatrix}
\end{equation*}
Tensor de inercia: 
\begin{align*}
I_c{1} & =\begin{bmatrix}
0 &	0 &	0 \\ 0 &	0 &	0 \\
0 &	0 &	0\end{bmatrix} \\
I_c{2} & =\begin{bmatrix}
0 &	0 &	0    \\
0  &	0 &	0 \\
0	& 0 &	0\end{bmatrix} \\
I_c{3} & =\begin{bmatrix}
0 &	0	& 0 \\0	0 & 0 \\
0 & 0 & 0\end{bmatrix}\\
I_c{4} & =\begin{bmatrix}
0 &	0 & 0 \\
0 & 0 &	0\\
0 & 0 &	0\end{bmatrix}\\
I_c{5} & =\begin{bmatrix}
0 &	0 & 0\\
0 &	0 & 0\\ 
0 & 0 & 0\end{bmatrix}\\
I_c{6} & =\begin{bmatrix}
0 &	0 & 0\\
0 &	0 & 0\\ 
0 & 0 & 0\end{bmatrix}
\end{align*}
Centros de Masa 
\begin{align*}
CM_{xi}=[0 & 0 & 0 & 0 & 0 & 0]\\
CM_{yi}=[0 & 0 & 0 & 0 & 0 & 0]\\
CM_{zi}=[0 & 0 & 0 & 0 & 0 & 0]\\
\end{align*}

\subsubsection{Matriz de Inercia}
La matriz de inercia del robot manipulador esta expresada por 
\begin{equation*}
    H=\begin{bmatrix} H_{11} & H_{12} & H_{13} & H_{14} & H_15 & H_{16}\\
    H_{21} & H_{22} & H_{23} & H_{24} & H_25 & H_{26}\\
    H_{31} & H_{32} & H_{33} & H_{34} & H_35 & H_{36}\\
    H_{41} & H_{42} & H_{43} & H_{44} & H_45 & H_{46}\\
    H_{51} & H_{52} & H_{53} & H_{54} & H_55 & H_{56}\\
    H_{61} & H_{62} & H_{63} & H_{64} & H_65 & H_{66}
\end{bmatrix}
\end{equation*}
\noindent Tomando en cuenta que cada $qi$ corresponde a $\theta_i$, $C_i$ corresponde a $\cos \theta_i$ y cada $S_i$ a
$\sin \theta_i$, cada elemento $H_{ij}$ de la matriz es representado por:

\subsubsection{Vector de Coriolis}
Para el robot dedo exoesqueleto propuesto, el vector de coriolis esta expresado por:
\begin{equation*}
    C=\begin{bmatrix} C_{11}\\
                    C_{21}\\
                    C_{31}\\
                    C_{41}\\
                    C_{51}\\
                    C_{61}\end{bmatrix}
\end{equation*}
\noindent Tomando en cuenta que cada $qi$ corresponde a $\theta_i$, $dqi$ corresponde a $\dot{\theta_i}$, $C_i$ corresponde
a $\cos \theta_i$ y cada $S_i$ a $\sin \theta_i$, cada elemento $C_{ij}$ de la matriz es representado por:


\subsubsection{Vector de gravedad}
El vector de gravedad final resultante de $\textbf{C}$, para el robot propuesto se representa por el siguiente vector
\begin{equation*}
    \textbf{g(q)}= \begin{bmatrix} g(q)_1 & g(q)_2 & g(q)_3 & g(q)_4 &g(q)_5 & g(q)_6  \end{bmatrix}
\end{equation*}
Tomando en cuenta que cada $C_1$ corresponde a $\cos \theta_i$ y cada $S_i$ a $\sin \theta_i$, cada  elemento $g(q)_i$ de
vector es representado por 