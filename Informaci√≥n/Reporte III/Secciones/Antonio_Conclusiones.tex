
\subsection{Luna Macías Antonio de Jesús}
    \noindent A partir del proyecto elaborado, se puede afirmar que se logró 
    de manera parcial el objetivo incial, referente al desarrollo 
    de un software que permitiera simular la dinámica de un exoesqueleto 
    actuado de cadena cinemática abierta con 6 GdL, esto por medio del desarrollo de 
    una aplicación de MATLAB integrada con bloques de Simulink.
    
    Por un lado, se implementaron de manera satisfactoria las ecuaciones 
    respectivas a la dinámica del robot por medio de la formulación de 
    D'Lambert-Lagrange, lo cual se reafirma con los resultados 
    obtenidos de posición, velocidad y análisis energético de las 
    pruebas realizadas para casos no conservativos, es decir 
    aquellos casos en los cuales se consideran los efectos de las fuerzas 
    disipativas. 

    Por otro lado, se observan discrepancias entre los resultados de análisis 
    energético para el caso conservativo, y los resultados esperados dada la conservación de 
    la energía. En este caso se observa en los primeros 5[s] de simulación un comportamiento 
    complementario entre las energías cinética y potencial, tendiendo a un valor 
    constante en la energía mecánica. El problema aparece cuando las velocidades 
    de la articulación $q_6$ incrementan, ya que a partir de ese instante de tiempo, se 
    visualiza por las gráficas que se genera más energía cinética de lo que el valor 
    máximo de energía potencial inicial debería permitir. De esta mnaera, se interpreta 
    que las discrepancias están dadas por un error acumulado entre el integrador 
    utilizado y el tiempo de muestreo, los cuales posiblemente generen una pérdida de información 
    en el proceso de integración.

    De esta manera, con respecto a los trabajos futuros, resalta el desarrollo de un 
    simulador basado en la teoría del BDA el cual permita reducir el orden de 
    complejidad de las Ecuaciones Diferenciales presentes en las matrices dinámicas, 
    y por lo tanto tender a la obtención de un simulador más eficiente y con 
    menor acumulación en el error.
