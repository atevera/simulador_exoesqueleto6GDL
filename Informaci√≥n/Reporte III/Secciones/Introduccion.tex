A lo largo del presente documento se describe el desarrollo matemático necesario para 
simular la dinámica directa de un exoesqueleto completamente actuado con 6 grados de 
libertad (GdL). El diseño del exoesqueleto se propone en el artículo original \cite{hexotrac} como un robot háptico 
colocado en el dedo índice, el cual permite simular las fuerzas generadas en un entorno virtual, 
para ser transferidas al usuario en el mundo real. 

De esta manera, el objetivo del proyecto es  desarrollar un simulador 
que permita observar el comportamiento de la dinámica directa del robot, así como analizar parámetros 
relacionados con el cambio sus coordenadas generalizadas y la energía total del sistema. Siedno la entrada del simulador los
torques aplicados en cada articulación, y la salida serán valores de posición y velocidad generalizadas variables en el tiempo.

Para esto, se explica el proceso de asignación de referenciales por medio de la metodología GRyMA \cite{rigid_multibody},
la obtención de las ecuaciones de cinemática directa a partir de los referenciales definidos en cada articulación, 
así como del modelo dinámico basado en la formulación de D'Lambert-Lagrange, considerando 
la aportación de las fuerzas exógenas, de gravedad, disipación y Coriolis. Además, se presenta la implementación de las 
ecuaciones en MATLAB y Simulink para la generación de una interfaz gráfica que permite presentar una visualización 
3D del robot, así como gráficas con las variaciones de movimiento y análisis energético.