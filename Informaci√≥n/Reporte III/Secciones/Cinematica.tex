\subsection{Cinemática directa}
    \subsubsection{Transformaciones homogéneas}
        De manera general, se explica una matriz de transformación homogénea como una
        matriz que permite expresar un punto especificado en coordenadas de
        un marco referencial con respecto a las coordenadas de otro marco 
        referencial, esto a partir de la aplicación de una rotación pura, 
        una traslación pura, o una combinación de ambas sobre el marco referencial
        inicial.

        De esta forma, se explica en \cite{3DMotion} que para cadenas cinemáticas abiertas, 
        donde el movimiento se encuentra restringido a una sola dirección, la 
        transformación homogénea desde el referencial padre hacia cualquier referencial 
        local se define únicamente por las coordenadas generalizadas escalares 
        correspondientes, definido de la siguiente manera.

        \begin{equation*} 
            A_i(q_i) \triangleq A^i_{pi}(q_i) =
            \begin{bmatrix}
                R^i_{pi}(q_i) & d^i_{pi}\\
                0 & 1
            \end{bmatrix}
            \epsilon SE(3)
        \end{equation*}

        Donde \emph{i} representa el marco referencial del cuerpo
        correspondiente, y \emph{pi} el marco referencial padre del 
        referencial \emph{i}.

        A partir de esta definición, se afirma que la transformación homogénea representa
        el movimiento rígido de un marco referencial dado con respecto a otro, y que además 
        presenta la propiedad de la propagación del Movimiento Rígido, que de manera resumida 
        permite realizar una multiplicación entre matrices de transformación homogénea de 
        marcos referenciales consecutivos, de tal manera que la matriz resultante representa 
        la transformación entre el primer y el último marco referencial, expresándose de la 
        siguiente manera.

        \begin{equation}
            A^i_0(q) =  A^{pi}_0(q_1,...,q_j)A_i(q_i)
            \label{eq:CD}
        \end{equation}

    \subsubsection{Asignación de referenciales}
        Con respecto al tema de transformaciones homogéneas, se observa que un 
        aspecto clave que se debe tener delimitado en el sistema es la localización 
        y orientación de los diferentes referenciales correspondientes a cada uno de 
        los cuerpos rígidos que lo conforman. En este caso, se busca tener un marco 
        referencial asigando a cada uno de los eslabones del exoesqueleto, con lo cual es 
        necesario pasar por un proceso de ubicación de referenciales en los diferentes 
        puntos de interés en el robot a analizar, a este proceso se le conoce como 
        asignación de referenciales.

        Como se explica en \cite{3DMotion} Existen diferentes metodologías para la asiganción 
        de referenciales en sistemas robóticos con cadenas cinemáticas abiertas, a 
        continuación se describirá de manera resumida el fundamento de las convenciones de 
        Denavit-Hartenberg (DH), que también aplican para su versión modificada (DHm), 
        al igual que se explicará la metodología propuesta por el Grupo de Robótica y 
        Manufactura Avanzada del CINVESTAV (GRyMA), siendo esta última la metodología 
        utilizada para la asignación de referenciales en el proyecto.

        Por un lado, se tiene que las convenciones \emph{DH} y \emph{DHm}, están conformadas 
        de algoritmos que delimitan la forma de establecer marcos de referencia en 
        cadenas cinemáticas donde los marcos consecutivos dependen únicamente de 4 parámtros, 
        en vez de 6 que sería el caso de considerar todas las posibles traslaciones y rotaciones 
        en los tres ejes \emph{[x,y,z]} del sistema coordenado. Para lograr dicha reducción de 
        parámetros, las metodologías hacen uso de 2 restricciones con respecto al movimiento de 
        un marco referencial a otro.

        \begin{itemize}
            \item El eje \emph{x} de un referencial dado debe ser perpendicular al eje \emph{z} de los marcos referenciales consecutivos.
            \item El eje \emph{x} de un referencial dado y el eje \emph{z} del referencial consecutivo deben intersectar.
        \end{itemize}

        De igual manera, se asume que el movimiento de cualquier marco de referencia con respecto 
        a su referencial padre, ya sea rotacional o prismático, debe realizarse siempre con respecto 
        al eje \emph{z} del referencial padre.

    \subsubsection{Cinemática GRyMA}
        Con respecto a la metodología GRyMA, se explica en \cite{3DMotion} que el proceso de asignación 
        de referenciales inicia con delimitar cada marco referencial $\Sigma_i$ a lo largo de los ejes 
        de las articulaciones que definen las coordenadas generalizadas $q_i$. Sin embargo, a 
        diferencia de las metodologías \emph{DH} y \emph{DHm}, en este caso no existe la 
        restricción de colocar el eje \emph{z} en la misma dirección que la dirección de movimiento de las 
        articulaciones, con lo cual la dirección de movimiento se define directamente con el 
        vector director extendido.

        \begin{equation*}
            \lambda_i = (\lambda^T_T, \lambda^T_R)^T \epsilon \mathbb{R}^6
        \end{equation*}

        De esta manera, cada uno de los marcos referenciales se puede colocar en la misma 
        orientación que el referencial base, con lo cual solamente resultan necesarios 3 
        parámetros de ajuste para determinar la distancia relativa entre el referencial 
        padre y cada referencial nominal en una posición de las coordenadas generalizadas de 
        \emph{\textbf{q} = 0}.

        \begin{equation*}
            d_{i0} = (d_{xi}, d_{yi}, d_{xi})
        \end{equation*}

        Con lo cual, considerando que se cumple la condición de tornillo explicada en \cite{3DMotion}, 
        $\lambda_{Ti} X \lambda_{Ri} = 0$, debido a que cada coordenada generalizada 
        induce ya sea un movimiento prismático o de revoluta, pero no una combinación de ambos, 
        entonces la matriz de trasformación homogénea entre dos referenciales se 
        expresa de la siguiente manera.
        
        \begin{equation}
            A_i(q_i) = A_{i0} A_{iv}(q_i) = 
            \begin{bmatrix}
                e^{[\lambda_{R_i}\times]q_i} & d_{i0} + \lambda_{Ti}q_i\\
                0 & 1
            \end{bmatrix}
            \label{eq:TH_GRYMA}
        \end{equation}

        Considerando para el proceso de implementación, que el Operador Producto Cruz 
        [\textbf{a}X] representa el producto cruz vectorial como una expresión 
        de una matriz simétrica sesgada de la siguiente forma.

        \begin{equation*}
            [\mathbf{a}\times] = 
            \begin{bmatrix}
                0 & -a_z & a_y\\
                a_z & 0 & -a_z\\
                -a_y & a_x & 0
            \end{bmatrix}
            \epsilon \mathbb{R}^{3x3}
        \end{equation*}

        Así, si todos los posibles movimientos se encuentran alineados con alguno de los 
        ejes principales de cada marco referencial, el vector director cinemático $\lambda_i$ 
        puede ser codificado con un parámetro escalar único $\Theta$, incluyendo el caso constante, 
        es decir el caso en el que no existe un movimiento variable en el tiempo.

        \begin{table}[H]
            \centering
                \begin{center}
                    \begin{tabular}{cccccccc}
                        $\Theta_i$ & 0 & 1 & 2 & 3 & 4 & 5 & 6\\
                        \hline \hline 
                        $\lambda_{T_i}$ & 0 & 1 & 0 & 0 & 0 & 0 & 0\\ 
                        $\lambda_{T_i}$ & 0 & 0 & 1 & 0 & 0 & 0 & 0\\
                        $\lambda_{T_i}$ & 0 & 0 & 0 & 1 & 0 & 0 & 0\\
                        \hline 
                        $\lambda_{R_i}$ & 0 & 0 & 0 & 0 & 1 & 0 & 0\\
                        $\lambda_{R_i}$ & 0 & 0 & 0 & 0 & 0 & 1 & 0\\
                        $\lambda_{R_i}$ & 0 & 0 & 0 & 0 & 0 & 0 & 1\\
                        \hline 
                        $R_{i}$ & $I_3$ & $I_3$ & $I_3$ & $I_3$ & $R_{x}(qi)$ & $R_{y}(qi)$ & $R_{z}(qi)$\\
                    \end{tabular}
                \end{center}
        \end{table}

        Con lo cual, la metodología GRyMA haría uso de 4 parámetros básicos constantes 
        para cada relación padre/hijo entre referenciales: $d_{x_1}, d_{y_1}, d_{z_1}$ 
        y $\Theta_i$.

        De otra manera, cada vector director de traslación y rotación puede ser parametrizado 
        con un par de elevación-azimtuh $(\alpha_i,\beta_i)$ de la siguiente manera, 
        aumentando el número de parámetros requeridos en 2. 

        \begin{table}[H]
            \centering
            \begin{center}
                \begin{tabular}{ccc}
                    $\Theta_i$ & 7 & 8\\
                    \hline \hline 
                    $\lambda_{T_i}$ & $\cos{\alpha_i}\sin{\beta_i}$ & 0\\ 
                    $\lambda_{T_i}$ & $\sin{\alpha_i}\sin{\beta_i}$ & 0\\
                    $\lambda_{T_i}$ & $\cos{\beta_i}$ & 0\\
                    \hline 
                    $\lambda_{R_i}$ & 0 & $\cos{\alpha_i}\sin{\beta_i}$\\
                    $\lambda_{R_i}$ & 0 & $\sin{\alpha_i}\sin{\beta_i}$\\
                    $\lambda_{R_i}$ & 0 & $\cos{\beta_i}$\\
                    \hline 
                    $R_{i}$ & $I_3$ & $R_{\lambda_{R_i}}$ (qi)\\ 
                \end{tabular}
            \end{center}
        \end{table}

    \subsubsection{Jacobiano geométrico}
        Se explica en \cite{3DMotion} que el Jacobiano Geométrico de un marco referencial 
        específico, es el operador lineal que permite mapear las velocidades generalizadas 
        $\textbf{\dot{q}}$ a los valores del \emph{twist} $v_i$, el cual es el vector que 
        contiene la velocidad lineal y angular de un marco referencial, y se define de la 
        siguiente manera. 

        \begin{equation*}
            \mathcal{V}  \triangleq 
            \begin{pmatrix}
                v \\
                \omega
            \end{pmatrix}
            \epsilon \mathcal{M} \subset \mathbb{R}^6
        \end{equation*}

        Y con lo cual, al ser relacionado con el Jacobiano geométrico, se obtiene la siguiente 
        igualdad.

        \begin{equation*}
            \mathcal{V} = J_i(q)\dot{q}
        \end{equation*}

        De esta manera, se observa que al ser el \emph{twist} un vector que puede ser expresado 
        en las coordenadas de cualquier marco referencial, entonces también es posible expresar 
        el Jacobiano geométrico de diferentes formas, en este caso surgen las siguientes expresiones 
        para la representación del Jacobiano en coordenadas inerciales y locales.

        \begin{align*}
            \mathcal{V}^{(0)}_i = \prescript{0}{}J_i(q)\dot{q}
            \mathcal{V}^{(i)}_i = \prescript{i}{}J_i(q)\dot{q}
        \end{align*}
        
        Así mismo, debido a la definición que tiene el vector \emph{twist}, se observa que de igual 
        manera el Jacobiano geométrico puede ser descompuesto en sus elementos referentes a velocidad 
        lineal y velocidad angular de la siguiente manera.

        \begin{equation*}
            \prescript{j}{}J_i(q) = 
            \begin{bmatrix}
                \prescript{j}{}J_{vi}(q) \\
                \prescript{j}{}J_{\omega i}(q)
            \end{bmatrix}
        \end{equation*}

        De tal manera que se tienen las siguientes equivalencias para la velocidad lineal y angular del 
        referencial local analizado.

        \begin{align*}
            v^{(j)}_i = \prescript{j}{}J_{vi}(q)\dot{q}
            \omega^{(j)}_i = \prescript{j}{}J_{\omega i}(q)\dot{q}
        \end{align*}

        Cabe mencionar que los componentes de velocidad del Jacobiano pueden ser expresados 
        con respecto al referencial inercial, solamente es necesario hacer la multiplicación 
        por la matriz de transformación homogénea respectiva del referencial inercial al referencial 
        correspondiente del Jacobiano local.

        A continuación se presentarán las ecuaciones utilizadas en la implementación para el 
        cálculo de los componentes de velocidad del Jacobiano geométrico.

    \subsubsection{Jacobiano geométrico de velocidad lineal}
        Como primer punto se explica en \cite{3DMotion} que el Jacobiano de velocidad lineal 
        en una cadena serial, puede ser expresado de la siguiente manera.

        \begin{equation}
            \prescript{0}{}J_{Vi}(q) = 
            [[\prescript{0}{}J_{Vi}]_1, [\prescript{0}{}J_{Vi}]_2, ..., \lambda^{0}_{Ti}, 0, ..., 0]
            \epsilon \mathbb{R}^{3xn}
            \label{eq:Jv}
        \end{equation}

        Donde cada columna del operador esta dado por la ecuación siguiente.

        \begin{equation*}
            [\prescript{0}{}J_{Vi}]_k = \frac{\partial v^{0}_i}{\partial \dot{q}k} = 
            \begin{cases}
                \lambda^{(0)}_{Tk} + \lambda^{(0)}_{Rk} \times (d_i - d_k) & si k \leqq i \\
                0                                                          & Otro
            \end{cases}
        \end{equation*}

    \subsubsection{Jacobiano geométrico de velocidad angular}
        Como segundo punto, se explica que el Jacobiano de velocidad angular, 
        en cadenas seriales, puede ser calculado para cualquier marco de referencia 
        haciendo uso de los vectores directores correspondientes a los referenciales 
        subsecuentes, obteniéndose la siguiente expresión.

        \begin{equation}
            \prescript{0}{}J_{\omega i} =
            [\lambda^{0}_{R1}, \lambda^{0}_{R2}, ..., \lambda^{0}_{Ri}, 0, ..., 0]
            \epsilon \mathbb{R}^{3xn}
            \label{eq:Jw}
        \end{equation}

        Esto es condierando que a partir de la metodología GRyMA para asignación de referenciales, 
        el vector director de rotación con respecto al referencial inercial se expresa de la siguiente 
        forma.

        \begin{equation*}
            \lambda^{0}_{Ri}(\textbf{q}) = R^i_0(q)\lambda_{R_i}
        \end{equation*}