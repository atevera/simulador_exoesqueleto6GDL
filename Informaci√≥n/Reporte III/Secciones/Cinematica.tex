\subsection{Cinemática directa}
% Tony -------- Info
    \subsubsection{Transformaciones homogéneas}
        De manera general, se explica una matriz de transformación homogénea como una
        matriz que permite expresar un punto especificado en coordenadas de
        un marco referencial con respecto a las coordenadas de otro marco 
        referencial, esto a partir de la aplicación de una rotación pura, 
        una traslación pura, o una combinación de ambas sobre el marco referencial
        inicial.

        De esta forma, se explica en [Multirigid Body] que para cadenas cinemáticas abiertas, 
        donde el movimiento se encuentra restringido a una sola dirección, la 
        transformación homogénea desde el referencial padre hacia cualquier referencial 
        local se define únicamente por las coordenadas generalizadas escalares 
        correspondientes, definido de la siguiente manera.

        \begin{equation*} 
            A_i(q_i) \triangleq A^i_{pi}(q_i) =
            \begin{bmatrix}
                R^i_{pi}(q_i) & d^i_{pi}\\
                0 & 1
            \end{bmatrix}
            \epsilon SE(3)
        \end{equation*}

        Donde \emph{i} representa el marco referencial del cuerpo
        correspondiente, y \emph{pi} el marco referencial padre del 
        referencial \emph{i}.

        A partir de esta definición, se afirma que la transformación homogénea representa
        el movimiento rígido de un marco referencial dado con respecto a otro, y que además 
        presenta la propiedad de la propagación del Movimiento Rígido, que de manera resumida 
        permite realizar una multiplicación entre matrices de transformación homogénea de 
        marcos referenciales consecutivos, de tal manera que la matriz resultante representa 
        la transformación entre el primer y el último marco referencial, expresándose de la 
        siguiente manera.

        \begin{equation*}
            A^i_0(q) =  A^{pi}_0(q_1,...,q_j)A_i(q_i)
        \end{equation*}

    \subsubsection{Asignación de referenciales}
        Con respecto al tema de transformaciones homogéneas, se observa que un 
        aspecto clave que se debe tener delimitado en el sistema es la localización 
        y orientación de los diferentes referenciales correspondientes a cada uno de 
        los cuerpos rígidos que lo conforman. En este caso, se busca tener un marco 
        referencial asigando a cada uno de los eslabones del exoesqueleto, con lo cual es 
        necesario pasar por un proceso de ubicación de referenciales en los diferentes 
        puntos de interés en el robot a analizar, a este proceso se le conoce como 
        asignación de referenciales.

        Como se explica en [Multirigid Body] Existen diferentes metodologías para la asiganción 
        de referenciales en sistemas robóticos con cadenas cinemáticas abiertas, a 
        continuación se describirá de manera resumida el fundamento de las convenciones de 
        Denavit-Hartenberg (DH), que también aplican para su versión modificada (DHm), 
        al igual que se explicará la metodología propuesta por el Grupo de Robótica y 
        Manufactura Avanzada del CINVESTAV (GRyMA), siendo esta última la metodología 
        utilizada para la asignación de referenciales en el proyecto.

        Por un lado, se tiene que las convenciones \emph{DH} y \emph{DHm}, están conformadas 
        de algoritmos que delimitan la forma de establecer marcos de referencia en 
        cadenas cinemáticas donde los marcos consecutivos dependen únicamente de 4 parámtros, 
        en vez de 6 que sería el caso de considerar todas las posibles traslaciones y rotaciones 
        en los tres ejes \emph{[x,y,z]} del sistema coordenado. Para lograr dicha reducción de 
        parámetros, las metodologías hacen uso de 2 restricciones con respecto al movimiento de 
        un marco referencial a otro.

        \begin{itemize}
            \item El eje \emph{x} de un referencial dado debe ser perpendicular al eje \emph{z} de los marcos referenciales consecutivos.
            \item El eje \emph{x} de un referencial dado y el eje \emph{z} del referencial consecutivo deben intersectar.
        \end{itemize}

        De igual manera, se asume que el movimiento de cualquier marco de referencia con respecto 
        a su referencial padre, ya sea rotacional o prismático, debe realizarse siempre con respecto 
        al eje \emph{z} del referencial padre.

    \subsubsection{Cinemática GRyMA}
        Con respecto a la metodología GRyMA, se explica en [Multirigid Body] que el proceso de asignación 
        de referenciales inicia con delimitar cada marco referencial $\Sigma_i$ a lo largo de los ejes 
        de las articulaciones que definen las coordenadas generalizadas $q_i$. Sin embargo, a 
        diferencia de las metodologías \emph{DH} y \emph{DHm}, en este caso no existe la 
        restricción de colocar el eje \emph{z} en la misma dirección que la dirección de movimiento de las 
        articulaciones, con lo cual la dirección de movimiento se define directamente con el 
        vector director extendido.

        \begin{equation*}
            \lambda_i = (\lambda^T_T, \lambda^T_R)^T \epsilon \mathbb{R}^6
        \end{equation*}

        De esta manera, cada uno de los marcos referenciales se puede colocar en la misma 
        orientación que el referencial base, con lo cual solamente resultan necesarios 3 
        parámetros de ajuste para determinar la distancia relativa entre el referencial 
        padre y cada referencial nominal en una posición de las coordenadas generalizadas de 
        \emph{\textbf{q} = 0}.

        \begin{equation*}
            d_{i0} = (d_{xi}, d_{yi}, d_{xi})
        \end{equation*}

        Con lo cual, considerando que se cumple la condición de tornillo explicada en [Multirigid Body], 
        $\lambda_{Ti} X \lambda_{Ri} = 0$, debido a que cada coordenada generalizada 
        induce ya sea un movimiento prismático o de revoluta, pero no una combinación de ambos, 
        entonces la matriz de trasformación homogénea entre dos referenciales se 
        expresa de la siguiente manera.
        
        \begin{equation*}
            A_i(q_i) = A_{i0} A_{iv}(q_i) = 
            \begin{bmatrix}
                e^{[\lambda_{R_i}\times]q_i} & d_{i0} + \lambda_{Ti}q_i\\
                0 & 1
            \end{bmatrix}
        \end{equation*}

        Considerando para el proceso de implementación, que el Operador Producto Cruz 
        [\textbf{a}X] representa el producto cruz vectorial como una expresión 
        de una matriz simétrica sesgada de la siguiente forma.

        \begin{equation*}
            [\mathbf{a}\times] = 
            \begin{bmatrix}
                0 & -a_z & a_y\\
                a_z & 0 & -a_z\\
                -a_y & a_x & 0
            \end{bmatrix}
            \epsilon \mathbb{R}^{3x3}
        \end{equation*}

        Así, si todos los posibles movimientos se encuentran alineados con alguno de los 
        ejes principales de cada marco referencial, el vector director cinemático $\lambda_i$ 
        puede ser codificado con un parámetro escalar único $\Theta$, incluyendo el caso constante, 
        es decir el caso en el que no existe un movimiento variable en el tiempo.

        \begin{table}[H]
            \centering
                \begin{center}
                    \begin{tabular}{cccccccc}
                        $\Theta_i$ & 0 & 1 & 2 & 3 & 4 & 5 & 6\\
                        \hline \hline 
                        $\lambda_{T_i}$ & 0 & 1 & 0 & 0 & 0 & 0 & 0\\ 
                        $\lambda_{T_i}$ & 0 & 0 & 1 & 0 & 0 & 0 & 0\\
                        $\lambda_{T_i}$ & 0 & 0 & 0 & 1 & 0 & 0 & 0\\
                        \hline 
                        $\lambda_{R_i}$ & 0 & 0 & 0 & 0 & 1 & 0 & 0\\
                        $\lambda_{R_i}$ & 0 & 0 & 0 & 0 & 0 & 1 & 0\\
                        $\lambda_{R_i}$ & 0 & 0 & 0 & 0 & 0 & 0 & 1\\
                        \hline 
                        $R_{i}$ & $I_3$ & $I_3$ & $I_3$ & $I_3$ & $R_{x}(qi)$ & $R_{y}(qi)$ & $R_{z}(qi)$\\
                    \end{tabular}
                \end{center}
        \end{table}

        Con lo cual, la metodología GRyMA haría uso de 4 parámetros básicos constantes 
        para cada relación padre/hijo entre referenciales: %d_{x_1}, d_{y_1}, d_{z_1}% 
        y $\Theta_i$.

        De otra manera, cada vector director de traslación y rotación puede ser parametrizado 
        con un par de elevación-azimtuh $(\alpha_i,\beta_i)$ de la siguiente manera, 
        aumentando el número de parámetros requeridos en 2. 

        \begin{table}[H]
            \centering
            \begin{center}
                \begin{tabular}{ccc}
                    $\Theta_i$ & 7 & 8\\
                    \hline \hline 
                    $\lambda_{T_i}$ & $\cos{\alpha_i}\sin{\beta_i}$ & 0\\ 
                    $\lambda_{T_i}$ & $\sin{\alpha_i}\sin{\beta_i}$ & 0\\
                    $\lambda_{T_i}$ & $\cos{\beta_i}$ & 0\\
                    \hline 
                    $\lambda_{R_i}$ & 0 & $\cos{\alpha_i}\sin{\beta_i}$\\
                    $\lambda_{R_i}$ & 0 & $\sin{\alpha_i}\sin{\beta_i}$\\
                    $\lambda_{R_i}$ & 0 & $\cos{\beta_i}$\\
                    \hline 
                    $R_{i}$ & $I_3$ & $R_{\lambda_{R_i}}$ (qi)\\ 
                \end{tabular}
            \end{center}
        \end{table}

        Ejemplo de mayor que $>$

    \subsubsection{Jacobiano geométrico}
    \subsubsection{Jacobiano geométrico de velocidad lineal}
    \subsubsection{Jacobiano geométrico de velocidad angular}


    % Eliezer ----- Info

    \noindent La cinemática directa consiste en definir la posición y orientación del efector final, en función de las
    coordenadas generalizadas de cada articulación, con respescto a un marco de referencia [3], en este caso particular,
    es un marco de referencia no inercial.

    Para lograrlo, se requieren cadenas cinemáticas desde el referencial base hasta el referencial local; para cada uno de
    estos ejes coordenados, se requiere el cálculo de matrices de transformación homogéneas mismas que definen el movimiento
    de rotación y traslación en función de donde se ubique el marco referencial de cada articulación. 

    Debido a que cada eslabón tiene asignada una coordenada generalizada y por lo tanto, tiene su propia matriz que describe
    su movimiento, se necesita multiplicar cada una de las matrices de transformación homogéneas.
    \\
    La matriz de transformación homogenea tiene la siguiente forma:
    \begin{equation*} 
        A_i = 
        \begin{bmatrix}
        R^i_{i-1} & d^i_{i-1}\\
        0 & 1
        \end{bmatrix}
    \end{equation*}

    \noindent Cabe destacar que la matriz $R^j_i \in \mathbb{R}^{3\times 3}$ representa la orientación del referencial $j$
    respecto al referencial $i$, y el vector $d^j_{i}$ expresa la traslación del referencial $j$ respecto al referencial $i$.\\

    \noindent Existen diferentes métodos para asignar los referenciales de coordenadas generalizadas, y así obtener la matriz
    de transformación homogenea; uno de los más utilizados en robótica industrial es el denominado: "Denavit Hartenberg",
    pero no es el único. En el curso se presentan 2 más (Denavit Hartenberg modificado "mDH" y GRyMA), lo cierto es que los
    3 buscan reducir la cantidad de parámetros que se requieren para definir las transformaciones homogéneas entre marcos
    referenciales. 

    \noindent Los movimientos rígidos generales se definen mediante una traslación $d \in \mathbb{R}^3$ y una rotación
    $R \in SO(3)$. Por tanto, se componen de 6 GDL para cada par de marcos referenciales consecutivos. La asignación del marco
    para cada elemento rígido da como resultado todos los $A_i (q_i) \in SE(3)$ en el sistema.

    \noindent Dos de los ejemplos más comunes son las convenciones clásicas de Denavit-Hartenberg (DH) y las modificadas de
    Denavit-Hartenberg (mDH). Ambos reducen los parámetros requeridos de 6 a 4, lo que da como resultado la torsión del enlace
    ($\alpha$), la longitud del enlace ($a$), el ángulo de unión ($\theta$) y el desplazamiento del enlace ($d$).
    \noindent A continuación se presentarán las reglas que se deben llevar a cabo para la asignación de referenciales,
    correspondiente.

    \subsubsection{Denavit Hartenberg}
    \noindent Existen dos restricciones al momento de definir del marco de referencia: 
    \begin{enumerate}
        \item El eje $x$ de un referencial dado debe ser perpendicular al eje $z$ de los referenciales vecinos consecutivas. 
        \item El eje $x$ de un referencial dado y el eje $z$ del referencial vecino consecutivo deben intersectar. 
    \end{enumerate}

    \noindent Debido a que la convención DH define una secuencia intrínseca explícita:
    $\theta_i \rightarrow d_i \rightarrow a_i \rightarrow \alpha_i$, las transformaciones homogéneas $A_i (q_i)$,
    que representan el movimiento rígido correspondiente, se obtiene la siguiente forma particular:
    \begin{align*}
        A_i (q_i) & \triangleq A_R (R_{z,\theta}) A_T (d_i k) A_T (a_i i) A_R (R_{x,\alpha_i}) \\
        & = \left[  \begin{array}{cc}
            R_{i-1}^i (\cdot)  & d_{i/i-1}^{(i-1)} (\cdot) \\
            0 & 1  
    \end{array} \right]
    \end{align*}
    donde
    \begin{align*}
        R_{i-1}^i (\cdot) & = R_{z,\theta_i}  R_{x,\alpha_i} \\ 
        & = \left[  \begin{array}{ccc}
            \cos{\theta_i}  & -\sin{\theta_i}\cos{\alpha_i} & \sin{\theta_i} \sin{\alpha_i} \\
            \sin{\theta_i} &  \cos{\theta_i}\cos{\alpha_i} &
            -\cos{\theta_i}\sin{\alpha_i} \\
            0 & \sin{\alpha_i} & \cos{\alpha_i}
        \end{array} \right]
    \end{align*}

    \vspace{2mm}
    \begin{equation*}
    \begin{array}{ccc}
        d_{i/i-1}^{(i-1)} (\cdot) & = d_i k + a_i R_{z,\theta_i} R_{x,\alpha_i}  i 
        & = \left[  \begin{array}{c}
            a_i \cos{\theta_i} \\
            a_i \sin{\theta_i} \\
            d_i 
        \end{array} \right]
    \end{array}
    \end{equation*}

    \vspace{5mm}
    \noindent Además se sabe que solo uno de los parámetros DH es variable en el tiempo, por lo tanto existen
    variaciones en el ángulo de articulación o en el desplazamiento del enlace\\
    \vspace{-2mm}
    \begin{equation*}
        \begin{array}{cc}
            \theta_i (t) = \theta_{i_0} + q_i (t) & \text{Articulación Revoluta} (R) \\
            d_i (t) = d_{i_0} + q_i (t) & \text{Articulación presmática} (P)
        \end{array}
    \end{equation*}
    \vspace{1mm}
    \noindent La cinemática directa de un robot manipulador puede ser determinada por la multiplicación de todas
    las matrices $A_i (q_i)$ obtenidas por la tabla de los parámetros de DH. 

    \noindent Sabiendo que el exoesqueleto tiene asignados 6 coordenadas generalizadas $q_i$ se tendría que
    aplicar la multiplicación de la siguiente manera:
    \begin{equation*}
        % Revisar porque le moví a los subíndices jeje Atte. Alejandro :)
    A = A_1(q_1) \hspace{1mm} A_2(q_2) \hspace{1mm} A_3(q_3) \hspace{1mm} A_4(q_4) \hspace{1mm} A_5(q_5)
        \hspace{1mm} A_6(q_6) \hspace{1mm} A_7(q_7)
    \end{equation*}
    \noindent Sin embargo, esto no es del todo cierto, debido a que para lograr cumplir con las 2 condiciones que
    estipula la metodología, aunado a la forma que tiene el diseño de los eslabones 1 y 3, implica que se generen
    referenciales virtuales.     \\ 
    \noindent Después de comparar los resultados de la matriz homogenea obtenida por las 3 metodologías y comparando
    los resultados de las mismas, se optó por no utilizar la metodología tradicional y llevar acabo los cálculos con
    el siguiente método: \\ 
    \subsubsection{GRyMA}
    \noindent La metodología GRyMA (Nombre dado en honor al Grupo de Robótica y Manufactura Avanzada) es otra alternativa
    a la asignación de los marcos referenciales en una cadena cinemática. Busca un cambio de paradigma y no exige las
    restricciones de DH mencionadas en la metodología anterior. \\ 
    \noindent El origen de cada marco de referencia $ \sum_i$ se coloca a lo largo del eje de articulación que definirá
    la coordenada generalizada $ qi$. No obstante, el eje z no está restringido a estar a lo largo de esta coordenada
    y la dirección del movimiento se define directamente con el vector director extendido
    $ \lambda_i=(\lambda_{Ti}^T, \lambda_{Ri}^T)^T \epsilon \thinspace \mathbb{R}^6 $.  \\ 
    \noindent Todas las coordenada generalizada de referencia, son colocadas en la misma orientación de "CASA" logrando
    una configuración nula y 3 parámetros de compensación son definidos para representar la distancia relativa desde el
    origen del marco padre para cada marco nominal $ \mathbf{d}_{io}=(d_{xi}, d_{yi}, d_{zi})$ en la posición de "CASA"
    (q=0).

    La transformación homogenea correspondiente es $A_i$ y tiene la siguiente forma:

    $A_i(qi)=A_{io}A{iV}(qi) \hfill \Sigma_i \rightarrow \Sigma_{pi},$

    $A_{io}$ representa la transformación homogénea constante y $A_iv(qi)$ la transformación homogénea variante en el
    tiempo.
    \begin{equation*}
        \hspace{-10mm}
        A_{io} = \left[
            \begin{array}{cc}
                I_{3} & d_{io}\\
                0 & 1\\
            \end{array}\right]  \epsilon \thinspace SE(3) 
    \end{equation*} 

    donde 

    \begin{equation*}
        \hspace{-10mm}
        d_{io} = \left(
            \begin{array}{c}
                d_{xi}\\
                d_{yi}\\
                d_{zi}\\
            \end{array}\right) \epsilon \thinspace \mathbb{R}^3 
    \end{equation*} 

    Por otra parte:

    \begin{equation*}
        \hspace{-10mm}
        A_{vi} (qi(t)) \doteq \left[
            \begin{array}{cc}
                e^{[\lambda_{Ri}X]_{qi}} & \lambda_(T_i)q_i\\
                0 & 1\\
            \end{array}\right]  
    \end{equation*} 

    $$ = 
    \left\{\begin{matrix}
    A_R (R\lambda_{Ri,qi}) & si \thinspace qi \thinspace es \thinspace rotacional \thinspace(\lambda_T=0) \\ 
    A_T (\lambda_{Ti}qi) & si \thinspace qi \thinspace es \thinspace prismatica \thinspace(\lambda_R=0) \\ 
    \end{matrix}\right.
    $$
    Se caracteriza con la coordenada generalizada escalar $qi\thinspace \epsilon \thinspace \mathbb{R} $ y el
    vector director extendido unitario constante (vector director cinemático)
    %Revisar expresión
    $$\lambda_{i}^{(1)} \coloneqq   
    \begin{pmatrix} A_{Ti} \\ A_{Ri} \\ \end{pmatrix} \thinspace   \epsilon \thinspace \mathbb{R}^4 \thinspace
    \thinspace \Rightarrow \thinspace \thinspace \lambda_{Ti}x\lambda_{Ri}=0$$

    Entonces:

    $$A_{i}(q_{i})=A_{io}A_{iv}(q_{i})=\begin{bmatrix} e^{[\lambda_{Ri}X]}q_{i}& d_{io}+\lambda_{Ti}q_{i}\\ 0 & 1\end{bmatrix}$$

    $e^{[\lambda_{Ri}X]qi}$ es la matriz de rotación correspondiente la cual es un mapeo exponencial (Formula de Rodrigues)

    $$R_{\lambda_{i}\vartheta}=I+[\lambda x]s\vartheta+[\lambda x]^{2}\upsilon\vartheta=e^{[\lambda x]\vartheta}$$
    donde
    %Revisar expresión
    $$ \epsilon_{q_i} = \thinspace verseno  \thinspace\thinspace \coloneqq  1 - cos(q_{i})$$ 

    Si todos los movimientos posibles son alineados con uno de los ejes principales de cada marco, el vector director
    cinemático $\Theta_i$ se puede codificar con un parámetro escalar único i, como se muestra a continuación:

    \begin{table}[!ht] %[H]
    \centering
    \begin{center}
    \begin{tabular}{cccccccc}
    $\Theta_i$ & 0 & 1 & 2 & 3 & 4 & 5 & 6\\
    \hline \hline 
    $\lambda_{T_i}$ & 0 & 1 & 0 & 0 & 0 & 0 & 0\\ 
    $\lambda_{T_i}$ & 0 & 0 & 1 & 0 & 0 & 0 & 0\\
    $\lambda_{T_i}$ & 0 & 0 & 0 & 1 & 0 & 0 & 0\\
    \hline 
    $\lambda_{R_i}$ & 0 & 0 & 0 & 0 & 1 & 0 & 0\\
    $\lambda_{R_i}$ & 0 & 0 & 0 & 0 & 0 & 1 & 0\\
    $\lambda_{R_i}$ & 0 & 0 & 0 & 0 & 0 & 0 & 1\\
    \hline 
    $R_{i}$ & $I_3$ & $I_3$ & $I_3$ & $I_3$ & $R_{x}(qi)$ & $R_{y}(qi)$ & $R_{z}(qi)$\\ 

    \end{tabular}
    \end{center}
    \end{table}
    En caso contrario, cada $\lambda_{T_i}$ , $\lambda_{R_i}$ puede parametrizarse con un
    par elevación-azimut ($\alpha_i$, $\beta_i$):

    \begin{table}[!ht] %[H]
    \centering
    \begin{center}
    \begin{tabular}{ccc}
    $\Theta_i$ & 7 & 8\\
    \hline \hline 
    $\lambda_{T_i}$ & $\cos{\alpha_i}\sin{\beta_i}$ & 0\\ 
    $\lambda_{T_i}$ & $\sin{\alpha_i}\sin{\beta_i}$ & 0\\
    $\lambda_{T_i}$ & $\cos{\beta_i}$ & 0\\
    \hline 
    $\lambda_{R_i}$ & 0 & $\cos{\alpha_i}\sin{\beta_i}$\\
    $\lambda_{R_i}$ & 0 & $\sin{\alpha_i}\sin{\beta_i}$\\
    $\lambda_{R_i}$ & 0 & $\cos{\beta_i}$\\
    \hline 
    $R_{i}$ & $I_3$ & $R_{\lambda_{R_i}}$ (qi)\\ 
    \end{tabular}
    \end{center}
    \end{table}

    Por lo cual, la metodología GRyMA usa únicamente 4 parámetros constantes independientes básicos  
    para cada relación de los marcos padre/hijo:
    $d_{xi}, d_{yi}, d_{zi}$ y $\theta$

    Para la asignación de marcos se respeta el siguiente algoritmo:
    \begin{enumerate}
    \item Identificar los ejes de movimiento en cada articulación.
    \item Asignar el marco de referencia inercial $\Sigma_0$ de modo que tanto la posición como la orientación sean
    estratégicamente definidas con respecto a los ejes de articulación del sistema.
    \item Asignar cada marco de referencia $\Sigma_i$ a la articulación correspondiente con la misma orientación del
    marco inercial y con el origen a lo largo del eje de articulación.
    \item Determinar el vector de distancia di0 ∈ R3 desde el marco padre de cada unión, en la posición "home" ($q = 0$).
    \item Codificar el parámetro de dirección $\Theta _i$ con respecto a la dirección y el tipo de movimiento de cada
    articulación.
    
    \end{enumerate}

    \noindent Finalmente, aplicando lo anterior y asignando los marcos referenciales como se aprecia en la Figura
    \ref{fig:ExoPara}, se obtiene la siguiente tabla:

    \begin{table}[!ht] %[H]
    \centering
    \begin{center}
    \begin{tabular}{cccccc}
    $\Sigma_i$ & $\Sigma_{p_i}$ & $d_{x_i}$ & $d_{y_i}$ & $d_{z_i}$ & $\Theta_i$\\
    \hline \hline 
    $\Sigma_1$ & $\Sigma_0$ & 0   & 0 & 0  & 5\\ 
    $\Sigma_2$ & $\Sigma_1$ & L1  & 1 & L2 & 5\\
    $\Sigma_3$ & $\Sigma_2$ & -L7 & 0 & L3 & 8\\
    $\Sigma_4$ & $\Sigma_3$ & -L6 & 0 & L4 & 6\\
    $\Sigma_5$ & $\Sigma_4$ & 0   & 0 & L5 & 4\\
    $\Sigma_6$ & $\Sigma_5$ & 0   & 0 & 0  & 5\\
    $\Sigma_6$ & $\Sigma_6$ & -L8 & 0 & 0  & 0\\
    \end{tabular}
    \end{center}
    \end{table}
    \subsection{Jacobiano}
    \noindent Existen dos tipos de jacobiano: El jacobiano analítico y el jacobiano geométrico. Este último depende de
    la configuración del manipulador y representa la relación entre las velocidades de la articulación, la velocidad
    lineal y angular de efector final. 
    En contraparte, el jacobiana analítico es cuando el efector final se expresa con referencia a una representación
    mínima (ángulos de Euler) en el espacio operacional, y se calcula derivando la posición del efector final y su
    orientación con respecto a las variables de la articulación [6]. La matriz jacobiana de un manipulador robótico es
    una matriz de $6 \times N$, donde la velocidad articular $ \dot{q}$ es un vector $N$ y la velocidad espacial $v$
    es un vector $-6$. La velocidad espacial y la velocidad articular están relacionadas a través de la matriz
    jacobiana mediante la siguiente expresión:
    \vspace{-3mm}

    \begin{equation*}
        \nu = J_i (q)\dot{q}
    \end{equation*}
    En el caso de que el manipulador tenga 6 GDL, no hay ningún problema a la hora de analizar la matriz jacobiana,
    ya que será una matriz de $ 6 \times  6 $ (matriz cuadrada), por lo que los cálculos de determinantes y rangos
    se pueden realizar sin ningún problema. Sin embargo, como se mencionó anteriormente, el diseño propuesto se
    clasifica como un manipulador de menos de 6 GDL. Debido a esto, la matriz jacobiana resultante será una matriz
    de $ 6 \times 5 $.
    \vspace{2mm}
    De esta manera la matriz correspondiente a jacobiana se expresa como:
    \begin{equation*}
        \hspace{-10mm}
        J = \left[
            \begin{array}{ccccc}
                J_{1,1} & J_{1,2} & J_{1,3} & J_{1,4} & J_{1,5}\\
                J_{2,1} & J_{2,2} & J_{2,3} & J_{2,4} & J_{2,5}\\
                J_{3,1} & J_{3,2} & J_{3,3} & J_{3,4} & J_{3,5}\\
                J_{4,1} & J_{4,2} & J_{4,3} & J_{4,4} & J_{4,5}\\
                J_{5,1} & J_{5,2} & J_{5,3} & J_{5,4} & J_{5,5}\\
                J_{6,1} & J_{6,2} & J_{6,3} & J_{6,4} & J_{6,5}\\
            \end{array}\right] 
        \hspace{10mm}
    \end{equation*} 
    Donde los valores representados en los primeros 3 renglones, corresponden a la velocidad lineal y los siguientes
    3 a la velocidad angular. 
    Para el caso de la velocidad lineal, se parte de la ecuación:
    \begin{equation*}
        J_{v_i} q \triangleq = \frac{\partial d_i}{\partial q}
    \end{equation*}
    Tomando en cuenta esto, la matriz del jacobiana de la velocidad lineal $J_{v}$ para el robot manipulador propuesto
    resultará de:
    \begin{equation*}
        J_v = J_{v_1}(q_1) \hspace{1mm} J_{v_2}(q_2) \hspace{1mm} J_{v_3}(q_3) \hspace{1mm} J_{v_4}(q_4) \hspace{1mm} J_{v_5}(q_5)
    \end{equation*}
    Para el caso de la velocidad angular, se parte de la ecuación:
    \begin{equation*}
        J_{\omega _i}q \triangleq = [r1_i] \times\frac{\partial r1_i}{\partial q} + [r2_i] \times\frac{\partial r2_i}{\partial q}
                                    + [r3_i] \times\frac{\partial r3_i}{\partial q}
    \end{equation*}
    Tomando en cuenta esto, la matriz del jacobiana de la velocidad angular $J_{\omega}$ para el robot manipulador propuesto
    resultará de:
    \begin{equation*}
        J_\omega = J_{\omega_1}(q_1) \hspace{1mm} J_{\omega_2}(q_2) \hspace{1mm} J_{\omega_3}(q_3) \hspace{1mm} J_{\omega_4}(q_4) \hspace{1mm} J_{\omega_5}(q_5)
    \end{equation*}
