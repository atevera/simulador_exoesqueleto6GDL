\subsection{Dinámica}
    La dinámica estudia las causas del movimiento, esto a través del estudio de las fuerzas y torques y su efecto en
    el movimiento de los cuerpos (esta representada en expresiones de segundo orden). En robótica, cuando se habla de
    dinámica, se hace referencia a la relación existente entre el movimiento del robot y las fuerzas generalizadas sobre el. 
    El modelo de Euler-Lagrange esta representado por: 
    \begin{equation*}
        H(q)\ddot{q} + C(q, \dot{q}) \dot{q} + g(q) = \tau
    \end{equation*}
    Donde $H(q)$ es la matriz de inercia, $C(q, \dot{q})$ es la matriz de coriolis  y $g(q)$ es el vector de gravedad.
    De igual forma, $\tau$ es la entrada des sistema.\\ 
    \subsubsection{Matriz de Inercia}
    La matriz de inercia, proviene de la energía cinemática del sistema, se encuentra descrita por
    \begin{equation*}
        K = \frac{1}{2} \dot{q}^T H(q)\dot{q}
    \end{equation*}
    De la ecuación anterior sur la matriz de inercia, expresada por
    \begin{equation*}
        H(q) = \sum_{i=o} ^n \{ m_i ^0 J_{v_cm_i} ^T(q) + ^0J_{\omega_i} ^T (q) R_0 ^i(q) I_c ^i {R_0 ^i}^T (q) ^0 J_{\omega_i} (q) \}
    \end{equation*}
    donde $m_i$ es la masa de cada eslabón $J_{v_cm_i}^T(q)$ es la matriz jacobiana de la velocidad lineal transpuesta para
    el centro de masa $i$ dependiente de $q$, $J_{\omega_i} ^T (q)$ es la matriz jacobiana de la velocidad angular transpuesta
    de $i$  dependiente de $q$, $I_c ^i$ es la matriz del tensor de inercia de $i$ y $J_{\omega_i} (q)$ es la matriz jacobiana
    de la velocidad angular.
    La matriz del tensor de inercia expresa el centro de masa en un marco referencial de coordenadas cartesianas. Es definida por
    \begin{equation*}
        \textbf{I}_c \triangleq \int_B [r \times]^T [r \times] dm 
    \end{equation*}
    Dentro de las propiedades de la matriz, es que esta debe ser definida positiva. \\

    \subsubsection{Matriz de Coriolis}
    El vector de fuerzas centrípetas y de Coriolis están representadas por $C_{q}(q,\dot{q})\dot{q}$ 
    Como la matriz de Coriolis depende del vector de coordenadas generalizadas y las velocidades correspondientes, al
    multiplicarla con el vector de velocidades en coordenadas generalizadas, se obtienen velocidades cruzadas (cuadrática en velocidada)
    y debido a ello no hay una única matriz de Coriolis. 

    Calcularla con los símbolos de Christofel nos da la propiedad de anti-simetría con respecto a la matriz de inercia.

    $$\left | C_{q}(q,\dot{q}) \right |{k j}=\sum_{i=1}^{n}=c_{ijk}(q)\dot{q_{i}}$$

    donde: 

    $$c_{ijk}(q)=\frac{1}{2}\left \{ \frac{\delta h_{kj}(q)}{\delta q_{i}}+\frac{\delta h_{ik}(q)}{\delta q_{j}}
                 -\frac{\delta h_{ij}(q)}{\delta q_{k}} \right \}$$

    El vector de Coriolis se obtiene multiplicando la matriz de Coriolis por las velocidades en coordenadas generalizadas

    $$C_{q}(q,\dot{q})\dot{q}=\begin{pmatrix}
    \vdots\\
    \sum_{ij=1}^{n}=C_{ijk}(q)\dot{q_{i}}\dot{q_{j}}\\ 
    \vdots\\
    \end{pmatrix}
    =\begin{pmatrix}
    \vdots\\
    \dot{q}^{T}C_{k}(q)\dot{q}\\ 
    \vdots\\
    \end{pmatrix}$$

    \subsubsection{Vector de gravedad}
    El vector de gravedad representa los torques sobre el robot debido al peso de cada eslabón, para obtenerlo se parte
    de la energía potencial. La energía potencial de cada cuerpo viene dada por su peso multiplicado por la posición
    vertical de su centro de masa
    $$U_{i}=-m_{i}g_{0}\cdot d_{cmi}(q)=-m_{i}d^{T}_{cmi}(q)g{o'}$$
    $d_{cmi}$(q) ∈ R3 es la posición cartesiana del vector de masa del cuerpo $i$

    $g_0 ∈ R3$ es la expresión inercial para la dirección de la gravedad (se define de acuerdo con la definición de la
    referencia del marco inercial del sistema) y el signo menos es una corrección tal que la energía potencial aumenta
    en contra de la dirección de la gravedad.\\
    La potencia total es una función que varia solo en la configuración q del sistema: 
    $$g_{q}=-\sum_{i=1}^{N}m_{i}\frac{\delta d_{cmi}(q)^{T}}{\delta q}g_{0}=-\sum_{i=1}^{N}m_{i}J_{v_{cmi}}^{T}(q)g_{0}$$

    donde:

    $$d_{cmi}(q)=d_{i}+R^{i}_{0}(q)r_{ci}$$

    La gradiente de la energía potencial es calculada como:

    $$g_{q}=- \sum_{i=1}^{n}m_{i}\frac{\delta d_{cmi}(q)^{T}}{\delta q}g_{0}=-\sum_{i=1}^{n}m_{i}J_{v_{cmi}}^{T}(q)g_{0}$$