\subsection{Dinámica}
    La dinámica es la parte de la mecánica que estudia la relación entre el movimiento y las causas que lo producen
    (fuerzas o torques) mediante el análisis de ecuaciones diferenciales de segundo orden (modelo dinámico). Para su
    obtención, pueden utilizarse diferentes metodologías a partir de la mecánica Newtoniana o bien de la mecánica
    Lagrangiana. 
    
    Especialmente, para sistemas multicuerpos rígidos, se prefiere el uso de la mecánica Lagrangiana dada su versatilidad
    y fácil escalado mediante el análisis energético para sistemas con $n$ partículas. Además, reduce drásticamente el número de ecuaciones
    necesarias para describir el movimiento de un conjunto de partículas; ya que sólo necesitaremos $n$ ecuaciones y no $3n$ como es el caso 
    de la mecánica Newtoniana.    
    Para ello, el sistema debe poder ser descrito mediante un conjunto de coordenadas generalizadas $\boldsymbol{q} \in \mathbb{R}^n$ y sus 
    derivadas totales respecto al tiempo $\boldsymbol{\dot{q}}$ (ambas medibles); las cuales representan las direcciones del movimiento
    admisible del sistema. 
    
    \subsubsection{Ecuación de D'Alambert-Lagrange}
    De acuerdo a \cite{3DMotion}, si se considera el principio de \emph{trabajo virtual} orientado al equilibrio estático del principio de mínima accción, puede partirse a la
    construcción del \emph{principio de D'Alambert}; siendo una extensión y enfocado al equilibrio dinámico del sistema.  Lo que permite obtener la
    expresión (\ref{eqn:DL_Equation}) que representa la \emph{Ecuación de D'Alambert-Lagrange} de forma vectorial para $\boldsymbol{q}$ linealmente
    independientes.
    \begin{equation} 
        \label{eqn:DL_Equation}
         \frac{d}{dt} \frac{\partial K}{\partial \boldsymbol{\dot{q}}} - \frac{\partial K}{\partial \boldsymbol{q}} = \boldsymbol{Q}
         \in \mathbb{R}^n
    \end{equation}
    donde
    \begin{equation}
        \label{eqn:kinetic_energy}
         K = \frac{1}{2} \boldsymbol{\dot{q}}^T H(\boldsymbol{q}) \boldsymbol{\dot{q}}
    \end{equation}
    define la energía cinética total del sistema relacionada con la matriz de inercia $H(\boldsymbol{q})$ y 
    \begin{equation}
        \label{eqn:fuerzas_generalizadas}
         \boldsymbol{Q} \triangleq \begin{bmatrix} Q_1 \\ \vdots \\ Q_n \end{bmatrix}
    \end{equation}
    representa las \emph{fuerzas generalizadas}
    relacionadas a la suma de las fuerzas efectivas $\boldsymbol{f}_{e_j}$ que experimenta cada cuerpo $j$ respecto a la coordenada generalizada $q_i$
    como resultado de los siguientes efectos:
    \begin{itemize}
        \item Potenciales conservativos
        \item Disipación (para sistemas no conservativos)
        \item Restricciones (no necesariamente \emph{holonómicas})
        \item Fuerzas exógenas (normalmente definidas por el usuario)
    \end{itemize}

    \subsubsection{Fuerzas Inerciales}   
    Con base en \cite{3DMotion}, cada conjunto de partículas tiene una \emph{fuerza generalizada de inercia} $\boldsymbol{\tau}_I$ que representa la compensación
    virtual del movimiento definido como el valor negativo de (\ref{eqn:DL_Equation}). 
    \begin{equation}
        \label{eqn:inertia_general_force}
        -\boldsymbol{\tau}_I \triangleq \frac{d}{dt} \frac{\partial K}{\partial \boldsymbol{\dot{q}}} - \frac{\partial K}{\partial \boldsymbol{q}} \in \mathbb{R}^n
    \end{equation}
    
    Al resolver (\ref{eqn:inertia_general_force}) en términos de (\ref{eqn:kinetic_energy}), se obtiene:
    \begin{equation}
        \label{eqn:inertial_terms}
        -\boldsymbol{\tau}_I = H(\boldsymbol{q}) \boldsymbol{\ddot{q}} + \dot{H}(\boldsymbol{q}, \boldsymbol{{\dot{q}}}) \boldsymbol{{\dot{q}}}
        - \frac{1}{2} \frac{\partial \left \{ \boldsymbol{\dot{q}}^T H(\boldsymbol{q}) \boldsymbol{\dot{q}} \right \} }{\partial \boldsymbol{q}}
    \end{equation}
    Donde la suma del segundo y tercer término representan al vector de Coriolis
    \begin{equation}
        \label{eqn:coriolis_term}
        C(\boldsymbol{q}, \boldsymbol{\dot{q}}) \boldsymbol{\dot{q}} = \dot{H}(\boldsymbol{q}, \boldsymbol{{\dot{q}}}) \boldsymbol{{\dot{q}}}
        - \frac{1}{2} \frac{\partial \left \{ \boldsymbol{\dot{q}}^T H(\boldsymbol{q}) \boldsymbol{\dot{q}} \right \} }{\partial \boldsymbol{q}}
    \end{equation} 
    Por lo tanto, la expresión (\ref{eqn:DL_Equation}) puede reescribirse como:
    \begin{equation}
        \label{eqn:DL_vectors}
        H(\boldsymbol{q}) \boldsymbol{\ddot{q}} + C(\boldsymbol{q}, \boldsymbol{\dot{q}}) \boldsymbol{\dot{q}} = \boldsymbol{Q}
    \end{equation}
    
    \subsubsection{Matriz de Inercia}
    Considerando una partícula $i$ de masa constante $m$, su energía cinética es proporcional a su velocidad traslacional $v_i$ 
    y rotacional $\omega_i$ en el centro de masa en coordenadas inerciales \cite{theoretical_minimun}.
    \begin{equation}
        \label{eqn:cinetica_normal}
         K_i = \frac{1}{2} m_i {v_i^{(0)}}^2 + \frac{1}{2} I_{c_i}^{(0)} {\omega_i^{(0)}}^2 
    \end{equation}
    Para sistemas multipartículas \cite{rigid_multibody}, 
    \begin{equation}
        \label{eqn:cinetica_multi}
         K = \sum_{i=1}^n K_i
    \end{equation}
    Por lo que si (\ref{eqn:cinetica_normal}) es equivalente a (\ref{eqn:kinetic_energy}), entonces la matriz de inercia $H(\boldsymbol{q})$ debe 
    contener la misma información que los términos de (\ref{eqn:cinetica_normal}). 
    
    Para ello, es necesario el uso de (\ref{eq:Jw}) para el mapeo de las velocidades lineales y angulares de los centros de masa
    de cada eslabón en función de las coordenadas generalizadas. Obteniendo: 
    \begin{multline}
        \label{eqn:inertia_matrix}
        H(\boldsymbol{q}) = \sum_{i=1}^n \{ m_i \: ^0J_{v{cm_i}}^T(\boldsymbol{q}) \: ^0J_{v{cm_i}}(\boldsymbol{q}) \\ 
        + ^0J_{\omega_{cm_i}}^T (\boldsymbol{q}) \: R_0^i(\boldsymbol{q}) \: \boldsymbol{I}_c^{(i)} \: {R_0^i}^T (\boldsymbol{q}) \: {^0J_{\omega_{cm_i}}}(\boldsymbol{q}) \}
    \end{multline}
    donde 
    \begin{equation}
        \label{eqn:tensor_inercia}
        \boldsymbol{I}_c \triangleq - \int_B [\boldsymbol{r} \times]^2 dm  = 
        \begin{bmatrix}
            I_{xx_c} & I_{xy_c} & I_{xz_c} \\
            I_{xy_c} & I_{yy_c} & I_{yz_c} \\
            I_{xz_c} & I_{yz_c} & I_{zz_c} \\
        \end{bmatrix}
    \end{equation}
    es el tensor de inercia en el centro de masa de un cuerpo $B$, siendo $\boldsymbol{r}$ el vector de posición del centro de masa respecto a su referencial local no inercial.
    
    De acuerdo a (\ref{eqn:tensor_inercia}), la matriz es simétrica y está compuesta por los momentos de inercia (en su diagonal principal) y los productos de inercia 
    (para los elementos fuera de ella). 

    Por otra parte, la matriz de inercia $H(\boldsymbol{q})$ cuenta con ciertas propiedades:
    \begin{itemize}
        \item Es simétrica $H(\boldsymbol{q}) = H(\boldsymbol{q})^T$
        \item Es definida positiva $H(\boldsymbol{q})=0$ \textbf{revisaar}
    \end{itemize}

    \subsubsection{Vector de Coriolis}
    $C(\boldsymbol{q}, \boldsymbol{\dot{q}}) \boldsymbol{\dot{q}}$ representa las fuerzas centrípetas y de Coriolis 
    en función a las variables de estado del sistema ($\boldsymbol{q}, \boldsymbol{\dot{q}}$). 
    
    Es importante recalcar que $C(\boldsymbol{q}, \boldsymbol{\dot{q}}) \in \mathbb{R}^{nxn}$ es la matriz de Coriolis y puede tener diferentes valores para el mismo robot, 
    mientras que $C(\boldsymbol{q}, \boldsymbol{\dot{q}}) \boldsymbol{\dot{q}} \in \mathbb{R}^n$ es único.

    El vector de Coriolis puede ser expresado como:
    \begin{equation}
        \label{eqn:coriolis_vector}
        C(\boldsymbol{q}, \boldsymbol{\dot{q}}) \boldsymbol{\dot{q}} = \begin{bmatrix} \vdots\\
        \sum_{i,j}^{n} c_{ijk}(\boldsymbol{q})\dot{q_{i}}\dot{q_{j}}   \\  \vdots\\ \end{bmatrix}
    \end{equation}
    donde 
    \begin{equation}
        \label{eqn:christoffel}
        c_{ijk}(q) \triangleq \frac{1}{2}\left( \frac{\partial h_{kj}(\boldsymbol{q})}{\partial q_{i}}+\frac{\partial h_{ik}(\boldsymbol{q})}{\partial q_{j}}
        -\frac{\partial h_{ij}(\boldsymbol{q})}{\partial q_{k}} \right)
    \end{equation}
    son los \emph{Símbolos de Christoffel}. Donde $h$ es el elemento en la matriz de inercia, $k$ corresponde a la posición en el vector de Coriolis; 
    mientras que $i$ y $j$ permiten obtener el producto de las velocidades generalizadas. 

    Aunque el enfoque es el vector de Coriolis, resulta importante considerar la propiedad \emph{Skew-Symmetry}. De acuerdo a [@Olguin_Multibody],
    se establece las siguientes relaciones con la matriz de inercia $H(\boldsymbol{q})$:
    \begin{equation}
        \label{eqn:skew1}
        C(\boldsymbol{q}, \boldsymbol{\dot{q}}) - \frac{1}{2}\dot{H}(\boldsymbol{q}) = Q 
    \end{equation}
    \begin{equation}
        \label{eqn:skew2}
        Q + Q^T = 0
    \end{equation}
    \begin{equation}
        \label{eqn:skew3}
        C(\boldsymbol{q}, \boldsymbol{\dot{q}}) + C^T(\boldsymbol{q}, \boldsymbol{\dot{q}}) = \dot{H}(\boldsymbol{q})
    \end{equation}

    \subsubsection{Vector de disipación}
    Como se presentó en las propiedades de la expresión (\ref{eqn:fuerzas_generalizadas}), las fuerzas generalizadas pueden descomponerse en (\ref{eqn:torques_generalizados}).
    \begin{equation}
        \label{eqn:torques_generalizados}
         \boldsymbol{Q} = \boldsymbol{\tau}_U + \boldsymbol{\tau}_D + \boldsymbol{\tau}_C + \boldsymbol{\tau}
    \end{equation}
    Donde $\boldsymbol{\tau}_U$ corresponde al efecto de potenciales conservativos, $\boldsymbol{\tau}_D$ representa la disipación de energía, $\boldsymbol{\tau}_C$
    las fuerzas de contacto y $\boldsymbol{\tau}$ los torques aplicados en cada grado de libertad de acuerdo al usuario. 
    
    Si se consideran efectos de disipación y potencial conservativo en (\ref{eqn:DL_vectors}), puede obtenerse la siguiente expresión:
    \begin{equation}
        \label{eqn:DL_final}
        H(\boldsymbol{q}) \boldsymbol{\ddot{q}} + C(\boldsymbol{q}, \boldsymbol{\dot{q}}) \boldsymbol{\dot{q}} + D(\boldsymbol{q}, \boldsymbol{\dot{q}})
        + \boldsymbol{g}(\boldsymbol{q}) = \boldsymbol{\tau}
    \end{equation}
    Donde $D(\boldsymbol{q}, \boldsymbol{\dot{q}})$ es el vector de disipación y puede estar representado por una \emph{función de Rayleigh} tal como se presenta
    \begin{equation}
        \label{eqn:disipacion_ray}
        D(\boldsymbol{q}, \boldsymbol{\dot{q}}) = \frac{\partial \mathcal{R}}{\partial \boldsymbol{\dot{q}}}
    \end{equation} 
    O bien, por una fricción viscosa lineal $b_i$ proporcional a la velocidad generalizadas $\boldsymbol{\dot{q}}_i$ como:
    \begin{equation}
        \label{eqn:disipacion_simple}
        D(\boldsymbol{q}, \boldsymbol{\dot{q}}) = \begin{bmatrix} b_1 & 0 & 0 \\ 0 & \ddots & 0 \\ 0 & 0 & b_n  \end{bmatrix} 
    \end{equation}

    \subsubsection{Energía Potencial y Vector de gravedad}
    Para definir adecuadamente el vector de gravedad, es necesario considerar la definición matemática de la energía potencial $U$ para sistemas multipartículas con
    referencia a un \emph{datum} o valor de referencial $h_0$. Siendo
    \begin{equation}
        \label{eqn:energia_potencial}
         U_{h_0} = \sum_{i=1}^n U_{i_{h_0}} =  \sum_{i=1}^n m_i \boldsymbol{d}_{cm_i}^T \boldsymbol{g}_0
    \end{equation}
    donde $\boldsymbol{d}_{cm_i}$ corresponde al vector posición del centro de masa y $\boldsymbol{g}_0$ al vector de aceleración de la gravedad, ambos
    en coordenadas inerciales. 

    Si el sistema cuenta con valores negativos en la energía potencial, puede obtarse por sumar un valor constante $U_0$ con el fin de evitarlo y generar un \emph{offset}. 
    \begin{equation}
        \label{eqn:energia_potencial_offset}
         U = U_{h_0}+U_0 \: \: : \dot{U} = \dot{U}_{h_0} 
    \end{equation}

    Ya que la energía potencial gravitacional depende de las coordenadas generalizadas, el vector de gravedad $\boldsymbol{g}(\boldsymbol{q})$ puede
    expresarse como:
    \begin{equation}
        \label{eqn:vector_gravedad}
        \boldsymbol{g}(\boldsymbol{q}) = \frac{\partial U}{\partial \boldsymbol{q}}=-\sum_{i=1}^n \left[m_i \: ^0J_{v{cm_i}}^T(\boldsymbol{q}) \right] \boldsymbol{g}_0
    \end{equation}
    Donde el signo negativo es la correción tal que la energía potencial gravitacional aumenta en contra de la dirección de la gravedad.

    \subsection{Energía Mecánica}
    Tal como se expresa en \cite{theoretical_minimun},
    \begin{equation}
        \label{eqn:energia_mecanica}
         E = K + U
    \end{equation}
    es la energía mecánica del sistema.
    
    Para el caso conservativo, su valor debe ser constante ya que no existen pérdidas de energía o aumentos al ser un sistema ailsado y
    únicamente existirá transformación de energía potencial a cinética y viceversa. Por lo tanto, 
    \begin{equation}
        \label{eqn:derivada_energia_mecanica}
         \dot{E} = 0
    \end{equation}