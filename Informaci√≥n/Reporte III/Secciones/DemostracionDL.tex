Considerando la ecuación de movimiento de una partícula $j$ dentro de un sistema de $N$ partículas, dada la interacción
    con el resto del sistema. Es decir, la fuerza restrictiva $\boldsymbol{f_{r_j}}$ que restringe el movimiento de cada partícula $j$
    más la fuerza efectiva $\boldsymbol{f}_{e_j}$ es equivalente a la fuerza resultante $m_j \boldsymbol{\ddot{d}}_j$ dada
    la Segunda Ley de Newton. Visto de otro modo: 
    \begin{align}
        \label{eqn:DL_1particula}
         m_j \boldsymbol{\ddot{d}}_j - \boldsymbol{f}_{e_j} = \boldsymbol{f_{r_j}}
    \end{align}
    Aplicando el principio de trabajo virtual y el principio de D'Alambert, puede expresarse para un sistema multipartículas,
    también llamado \emph{Principio Generalizado de D'Alambert} como:
    \begin{align}
        \label{eqn:generalizado_dAlambert}
         \sum_{j=1}^{N} \left( \boldsymbol{f}_{e_j} + \boldsymbol{f_{r_j}}-m_j \boldsymbol{\ddot{d}}_j \right)
         \cdot \delta \boldsymbol{d}_j = 0
    \end{align}
    Donde el desplazamiento virtual Euclideano $\delta \boldsymbol{d}_j \in \mathbb{R}^3$ representa el \emph{movimiento
    admisible} de cada partícula en términos del desplazamiento virtual de cada coordenada generalizada $q_i$.
    \begin{align}
        \label{eqn:des_virtual_euclideano}
        \delta \boldsymbol{d}_j = \sum_{i=1}^N \frac{\partial \boldsymbol{d}_j }{\partial q_i} \delta q_i
    \end{align}
    Sustituyendo la expresión \ref{eqn:des_virtual_euclideano} en \ref{eqn:generalizado_dAlambert} y reordenando las
    sumatorias, se obtiene:
    \begin{align}
        \label{eqn:sum_DL}
         \sum_{i=1}^N \left[ \sum_{j=1}^N \left(m_j \boldsymbol{\ddot{d}}_j \cdot \frac{\partial \boldsymbol{d}_j }{\partial q_i}
         - \boldsymbol{f}_{e_j} \cdot \frac{\partial \boldsymbol{d}_j }{\partial q_i} \right) \right] \delta q_i = 0
    \end{align}
    Utilizando la propiedad de derivada de un producto y reorganizando, se construye:
    \begin{align}
        \label{eqn:sum2_DL}
         \sum_{i=1}^N \left[ \frac{d}{dt} \left( \sum_{j=1}^N m_j \boldsymbol{\dot{d}}_j \cdot \frac{\partial \boldsymbol{d}_j }{\partial q_i}\right)
         - \sum_{j=1}^N m_j \boldsymbol{\dot{d}}_j \cdot \frac{\partial \boldsymbol{\dot{d}}_j }{\partial q_i} - \sum_{j=1}^N
         \boldsymbol{f}_{e_j} \cdot \frac{\partial \boldsymbol{d}_j }{\partial q_i} \right] \delta q_i = 000
    \end{align}