\subsection{Cinemática directa}
    \noindent Considerando la asignación de los referenciales mediante la metodología GRyMA de la Figura \ref{fig:referencialesGRyMA},
    se presenta en la tabla siguiente los parámetros utilizados:

    \begin{table}[H]
        \centering
        \begin{center}
            \begin{tabular}{c||ccc|cc|c|c|c}
                \multirow{3}{*}{$q_i$} & \multicolumn{5}{c|}{Parámetros de localización}               & \multirow{3}{*}{$\Theta_i$} & \multirow{3}{*}{$\Sigma_i$} & \multirow{3}{*}{$\Sigma_{pi}$} \\
                                       & \multicolumn{3}{c|}{{[}mm{]}}  & \multicolumn{2}{c|}{{[}º{]}} &                             &                             &                                \\
                                       & $d_{xi}$ & $d_{yi}$ & $d_{zi}$ & $\alpha$      & $\beta$      &                             &                             &                                \\ \hline \hline
                $q_1$                  & 0        & 0        & 0        & -             & -            & 5                           & $\Sigma_1$                  & $\Sigma_0$                     \\
                $q_2$                  & 32.35    & 0        & 105.13   & -             & -            & 5                           & $\Sigma_2$                  & $\Sigma_1$                     \\
                $q_3$                  & -36.41   & 0        & 24.62    & 0º            & -59.26º      & 8                           & $\Sigma_3$                  & $\Sigma_2$                     \\
                $q_4$                  & -12.73   & 0        & 22.28    & -             & -            & 6                           & $\Sigma_4$                  & $\Sigma_3$                     \\
                $q_5$                  & 0        & 0        & 43.34    & -             & -            & 4                           & $\Sigma_5$                  & $\Sigma_4$                     \\
                $q_6$                  & 0        & 0        & 6        & -             & -            & 5                           & $\Sigma_6$                  & $\Sigma_5$                    
            \end{tabular}
        \end{center}
    \end{table}

    De esta forma, en la Sección \ref{cd:TH} se la función que construye  
    las matrices de transformación homogéneas entre marcos referenciales consecutivos, basado 
    en las expresiones (\ref{eq:TH_GRYMA}) y (\ref{eq:CD}), con el cual se obtiene como salida un arreglo de 6 matrices 
    expresable de la siguiente manera.

    \begin{equation*}
        A = [A^1_0(q_1), A^2_1(q_2), A^3_2(q_3), A^4_3(q_4), A^5_4(q_5), A^6_5(q_6)]
    \end{equation*}

    Posteriormente, se genera la función 
    de cinemática directa, la cual implementa la propiedad de propagación del movimiento 
    rígido, ya que permite obtener las matrices de transformación homogénea desde el marco 
    referencial inercial hasta cualquiera de los referenciales no inerciales por medio de su multiplicación. 

    El código de la función se puede visualizar en la sección \ref{cd:CD} donde su salida es un arreglo igualmente de 
    6 matrices al cual se le agregó el ajuste en coordenadas de posición correspondiente para 
    el análisis de los centros de masa de cada uno de los eslabones.

    \begin{equation*}
        A^i_0 = [A^1_0(q_1), A^2_0(q_1,q_2)...A^6_0(q_1,q_2,q_3,q_4,q_5,q_6)]
    \end{equation*}

    Finalmente, para la obtención de los Jacobianos geométricos de velocidad lineal y ángular, 
    se utilizan las expresiones (\ref{eq:Jv}) y (\ref{eq:Jw}). Permitiendo así el desarrollo del código para su
    generación como se expone en la Sección anexo \ref{cd:J}, el cual tiene como salida un arreglo de 6 matrices en
    representación del Jacobiano correspondiente a cada uno de los centros de masa.

    \textbf{¿Agregar vectores de rotación o Jacobianos?}


